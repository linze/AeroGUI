\section{Caso de uso general}
  \paragraph{Escenario principal de éxito}
  \begin{enumerate}
    \item El usuario accede al punto de venta.
    \item El punto de venta muestra una ventana con una descripción de sus servicios y preguntando el login.
    \item El usuario se logea.
    \item El punto de venta muestra un menú con las opciones disponibles.
    \item El usuario elige la opción que desea, si es que desea alguna.
    \item El punto de venta le da acceso a la opción deseada.
    \item El usuario concluye el uso de la opción deseada.
    \item El punto de venta vuelve al paso 6 hasta que no desea realizar más operaciones.
    \item El usuario cierra el punto de venta.
  \end{enumerate}
  \paragraph{Extensiones}
  \begin{enumerate}
    \item[2-3.] Si el usuario no dispone de credenciales, se le ofrece la capacidad de registrarse.
  \end{enumerate}

\section{FUN001: Consulta de viajes}
  \emph{Este caso de uso se enmarca en los pasos 7 y 8 del caso de uso general.}
  \paragraph{Escenario principal de éxito}
  \begin{enumerate}
    \item El usuario selecciona que desea realizar una consulta de viajes.
    \item El punto de venta le muestra las consultas que puede realizar.
    \item El usuario elige la opción que desea, si es que desea alguna.
    \item El punto de venta le da acceso a la consulta deseada.
    \item El usuario añade al carrito los productos que desee, indicando la cantidad.
    \item El usuario concluye el acceso a la consulta deseada.
    \item El punto de venta vuelve al paso 2 hasta que no desea realizar más consultas.
    \item El usuario cierra la vista de consultas.
  \end{enumerate}
  \paragraph{Extensiones}
  \begin{enumerate}
    \item[5.] Si no se dispone de asientos suficientes, se notifica el error.
  \end{enumerate}

  \subsection{CONS01: Consultas desde un origen}
    \emph{Este caso de uso se enmarca en los pasos 3 y 4 en caso de uso del requisito FUN001.}
    \paragraph{Escenario principal de éxito}
    \begin{enumerate}
      \item El usuario selecciona que desea consultar las tarifas desde un origen.
      \item El punto de venta da a elegir al usuario el origen de entre los que dispone información.
      \item El usuario selecciona un origen.
      \item El usuario pulsa el botón para realizar la consulta.
      \item El punto de venta muestra las tarifas para todos los transportes directos y todos los destinos desde el origen indicado.
      \item El usuario añade al carrito el trasnporte que desea.
      \item El punto de venta notifica al usuario de que este ha sido añadido.
      \item El usuario cierra la consulta.
    \end{enumerate}


    \paragraph{Extensiones}
    \begin{enumerate}
      \item[2.] En caso de no disponer ninguna información, se muestra un aviso al usuario y se va directamente al paso 6.
      \item[3.] Si el usuario no encuentra el origen que desea en la lista, va directamente al paso 6.
    \end{enumerate}

  \subsection{CONS02: Consulta de rutas entre dos ciudades}
    \emph{Este caso de uso se enmarca en los pasos 3 y 4 en caso de uso del requisito FUN001.}
    \paragraph{Escenario principal de éxito}
    \begin{enumerate}
      \item El usuario selecciona que desea consultar las rutas compuestas entre dos ciudades.
      \item El punto de venta da a elegir al usuario el origen y el destino de entre los que dispone información.
      \item El usuario selecciona un origen y un destino.
      \item El usuario pulsa el botón para realizar la consulta.
      \item El punto de venta muestra una lista con las diferentes rutas.
      \item El usuario pulsa en cada elemento de la lista para obtener más información al respecto.
      \item El usuario añade al carrito la ruta que desea.
      \item El punto de venta notifica al usuario de que esta ha sido añadida.
      \item El usuario cierra la consulta.
    \end{enumerate}

    \paragraph{Cuestiones abiertas}
    \begin{itemize}
       \item En los pasos 2 y 3. ¿Se da a elegir también la hora de salida o se buscan todas las posibilidades?
    \end{itemize}

  \subsection{CONS03: Consulta de tarifas de transportes directos para dos ciudades}
    \emph{Este caso de uso se enmarca en los pasos 3 y 4 en caso de uso del requisito FUN001.}
    \paragraph{Escenario principal de éxito}
    \begin{enumerate}
      \item El usuario selecciona que desea consultar las rutas directas entre dos ciudades.
      \item El punto de venta le da a elegir al usuario el origen y el destino entre los que dispone información.
      \item El usuario selecciona un origen y un destino.
      \item El usuario pulsa el botón para realizar la consulta.
      \item El punto de venta muestra la información sobre los trayectos directos.
      \item El usuario añade los que desee al carrito.
      \item El punto de venta notifica que han sido añadidos.
      \item El usuario cierra la consulta.
    \end{enumerate}

  \subsection{CONS04: Información de viajes concretos}
    \emph{Este caso de uso se enmarca en los pasos 3 y 4 en caso de uso del requisito FUN001.}
    \paragraph{Escenario principal de éxito}
    \begin{enumerate}
      \item El usuario selecciona que desea consultar información sobre un trayecto directo en concreto.
      \item El punto de venta le pide el usuario el identificador del transporte y trayecto.
      \item El usuario rellena los datos indicados.
      \item El punto de venta muestra los detalles (origen, destino, fecha y hora de la salida y la llegada y el estado) del transporte.
      \item El usuario cierra la ventana.
    \end{enumerate}
    \paragraph{Extensiones}
    \begin{enumerate}
      \item[4.] En caso de no disponer de datos sobre el identificador introducido se le notificará al usuario del error.
    \end{enumerate}



\section{FUN002: Reserva de viajes}
  \emph{Este caso de uso se enmarca en los pasos 7 y 8 del caso de uso general.}
  \paragraph{Escenario principal de éxito}
  \begin{enumerate}
    \item El punto de venta muestra una ventana con el contenido del carrito de compra.
    \item El punto de venta pide al usuario que acepte o cancele.
    \item El usuario acepta realizar la reserva del contenido mostrado.
    \item El punto de venta procede a reservar el contenido del carrito.
    \item El punto de venta notifica al usuario del éxito.
    \item El punto de venta cierra la ventana de reserva.
  \end{enumerate}
  \paragraph{Extensiones}
  \begin{enumerate}
    \item[5.] Si por el contrario el resultado ha sido un error, se le notifica al usuario de este y de la posible causa (sin entrar en detalles si esta fuese técnica).
  \end{enumerate}



\section{FUN003: Compra de billetes}
  \emph{Este caso de uso se enmarca en los pasos 7 y 8 del caso de uso general.}
  \paragraph{Escenario principal de éxito}
  \begin{enumerate}
    \item El usuario selecciona que desea abonar un viaje reservado.
    \item El punto de venta muestra una ventana donde se puede seleccionar la reserva que se desea abonar.
    \item El usuario selecciona la reserva y hace clic en el botón para realizar el pago.
    \item El punto de venta muestra una ventana preguntando el titular de la tarjeta, el número de esta y el código de verificación.
    \item El usuario introduce los datos y pulsa el botón para continuar.
    \item El punto de venta valida los datos.
    \item El punto de venta notifica que los billetes han sido abonados
    \item El punto de venta cierra las dos ventanas.
  \end{enumerate}
  \paragraph{Extensiones}
  \begin{enumerate}
    \item[7.] Si el billete no ha sido abonado se le informa al usuario de la causa del error (dando muy pocos detalles si es de carácter técnico) y se le permite modificar los datos introducidos.
  \end{enumerate}



\section{FUN004a: Login}
  \emph{Este caso de uso se enmarca en los pasos 2 y 3 del caso de uso general.}
  \paragraph{Escenario principal de éxito}
  \begin{enumerate}
    \item El punto de venta muestra una ventana con una descripción de sus servicios y pidiendo sus credenciales.
    \item El usuario introduce sus credenciales.
    \item El usuario pulsa el botón para logearse.
    \item El punto de venta comprueba que sus credenciales son correctas.
    \item El punto de venta muestra un menú con las opciones disponibles.
  \end{enumerate}
  \paragraph{Extensiones}
  \begin{enumerate}
    \item[4.] Si las credenciales son incorrectas, se notifica y se vuelve al paso 1.
  \end{enumerate}

\section{FUN004b: Registro}
  \emph{Este caso de uso se enmarca en los pasos 2 y 3 del caso de uso general.}
  \paragraph{Escenario principal de éxito}
  \begin{enumerate}
    \item El usuario pulsa el botón de registro.
    \item El punto de venta muestra una ventana pidiendo los datos personales (nombre, apellido y dirección) y los datos de acceso (correo electrónico y contraseña).
    \item El usuario intoduce los datos pedidos.
    \item El usuario pulsa el botón para validar el registro.
    \item El punto de venta valida que la dirección de correo electrónico es válida, que no está repetida y que las contraseñas coinciden.
    \item El punto de venta notifica al usuario de que se ha registrado correctamente en el sistema.
    \item El punto de venta cierra la ventana de registro.
  \end{enumerate}
  \paragraph{Extesiones}
  \begin{enumerate}
     \item[5.] Si alguna validación falla, se le notifica al usuario y se vuelve al paso 2, pero manteniendo los datos previamente introducidos.
  \end{enumerate}